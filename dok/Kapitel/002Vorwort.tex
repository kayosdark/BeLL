%Bibliographie
\section*{Bibliographische Beschreibung}
Bartel, Johann und Oehme, Peter\\\\
\q{\docTitle}\\\\
 \pageref{LastPage} Seiten, Y Anlagen, \totalfigures\ Abbildungen
 
  \newpage
 
 %ABSCHTRAKT
\section*{Erzeugung eines druckbaren 3D-Modells eines \\ Gebäudes anhand des Grundrisses}
 	
 
 Die Zielstellung dieser BeLL ist es, den Grundriss eines Hauses, der aus einem Konstruktionsprogramm entnommen wurde, in eine druckbare 3D-Datei zu konvertieren.
 Diese Umwandlung wird mithilfe eines Programmes mit eingebetteten selbst entworfenen mathematischen Operationen realisiert.\\\\
 Aus dem Grundriss, welcher eine 2D-Datenmenge darstellt, werden die digitalen Anweisungen für die 3D-Strukturen Wände, Grundflächen und Eckpfeiler berechnet. 
 Die Berechnungen laufen so ab, dass an allen Elementen komplementäre Stecker angebracht werden, die zusammen als ein Stecksystem fungieren. 
 Eckpfeiler dienen hierbei als Verbindungsstücke zwischen den Wänden und Bodenplatten die somit für die Stabilität des Objektes  sorgen. 
 Mit dem Stecksystem kann der gesamte 3D-Druck aus seinen einzelnen Bestandteilen aufgebaut werden. 
 Dadurch entsteht ein Modell, welches aufgrund der genannten Modifikationen transportabel und geeignet für Präsentationen ist.\\\\
 Immobilienkäufer und -verkäufer können dadurch die 3D-Darstellung der Immobilie  nutzen, um mehr Eindruck über das Objekt erlangen und um eine mögliche Inneneinrichtung zu planen.\\\\
 Johann Bartel und Peter Oehme
