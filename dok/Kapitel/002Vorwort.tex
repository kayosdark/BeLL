%Bibliographie
\chapter*{Bibliographische Beschreibung}
Bartel, Johann und Oehme, Peter\\\\
\q{\docTitle}\\\\
  \theseitennr\ Seiten, \totalfigures\ Abbildungen
 
  \newpage
 
 %ABSCHTRAKT
\chapter*{Erzeugung eines druckbaren 3D-Modells eines \\ Gebäudes anhand des Grundrisses}
Die Zielstellung dieser BeLL ist es, den Grundriss eines Hauses, der aus einem Konstruktionsprogramm entnommen wurde, in eine druckbare 3D-Datei zu konvertieren.
Diese Umwandlung wird mithilfe eines Programmes mit eingebetteten selbst entworfenen mathematischen Operationen realisiert.\\\\
Aus dem Grundriss, welcher eine 2D-Datenmenge darstellt, werden die digitalen Anweisungen für die 3D-Strukturen Wände, Grundflächen und \mbox{Eckpfeiler} berechnet. 
Diese Anweisungen lassen sich nach der Umwandlung in einem Modellierungsprogramm für den Druckvorgang umwandeln.
Die Berechnungen der Umwandlung laufen so ab, dass an allen Elementen des finalen Modells komplementäre Stecker angebracht werden, die zusammen als ein Stecksystem fungieren. 
Eckpfeiler dienen hierbei als Verbindungsstücke zwischen den Wänden und Bodenplatten, welche somit für die Stabilität des Objektes  sorgen. 
Das Stecksystem ermöglicht ein Zusammensetzen aller Bauteile zu einem stabilen Modell. 
Dadurch entsteht ein Modell, welches aufgrund der genannten Modifikationen transportabel und geeignet für Präsentationen ist.\\\\
Architekten können die 3D-Darstellung der Immobilie  nutzen, um mehr Eindruck über das Objekt zu erlangen und eine mögliche Inneneinrichtung zu planen.\\\\
Johann Bartel und Peter Oehme
