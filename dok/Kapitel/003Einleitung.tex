\chapter{Einleitung}
In den letzten Jahren gewannen 3D-Drucker immer mehr Bedeutung, sowohl für wissenschaftliche als auch für wirtschaftliche Zwecke. 
Sie werden genutzt, um verschiedene Gegenstände oder Bauteile des Eigenbedarfs selbst herzustellen oder nach Belieben anzupassen. 
Entsprechend naheliegend war es, dass schnell die ersten Modelle nachgebildet wurden, oder man sich an beliebten Steckbausteinsystemen wie LEGO orientierte, um sich eigene Sets zu drucken. \\
Diese Eignung für den Modellentwurf und Modellbau erweckte auch die Idee, ein Modell eines Hauses zu drucken, welches in sich aus strukturierten Bauteilen zusammengesetzt ist und somit auch das Entfernen einzelner dieser Bauteile erlaubt, um einen einfacheren Einblick in das Modell zu erhalten. 
Kombiniert mit dem Interesse an der Architektur entstand die Überlegung, ob es möglich wäre, anhand eines Grundrisses, welchen man aus einem Konstruktionsprogramm wie beispielsweise AutoCAD in Form einer .dxf-Datei erhalten kann, ein 3D-Modell des Hauses zu erzeugen, welches mithilfe eines Programmes automatisch in die vorgesehenen Bauteile zerlegt wurde, das im Anschluss von einem 3D-Drucker gedruckt werden kann.
 Dem Nutzer wird demnach nur zuteil, den Grundriss einzuspeisen und die ausgegebenen Bauteile korrekt auszudrucken,  was ihm einen aufwendigen Modellierungs- und Zerlegungsprozess erspart. \\
Ein solches Modell soll dann Architekten als Möglichkeit vorliegen, um ihren Kunden vor dem Kauf eines Hauses näheren Einblick in die Immobilie zu gewähren und mit ebenfalls 3D-gedruckten Möbeln bereits im Voraus erste Einrichtungsideen zu überprüfen. 
Diese Methode würde auf ein ausgeprägtes dreidimensionales Vorstellungsvermögen des Kunden verzichten und als Ergänzung zum vorgelegten Grundriss funktionieren.\\
Die Umsetzung des Programms erfolgt in der Programmiersprache Java.
Um die Problemstellung zu bewältigen, musste zunächst eine systematisch einzuhaltende Zerteilung des Modells festgelegt werden. 

