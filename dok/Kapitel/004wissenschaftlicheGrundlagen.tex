\chapter{Grundlagen}
\section{Planare Graphen}
Ein planarer, auch plättbarer Graph ist ein Graph der in einer Ebene mithilfe von Punkten bzw. Knoten (1) und Kanten (2) dargestellt werden kann, ohne dass sich zwei oder mehr Kanten schneiden. 
Jede Fläche (3) des Graphen wird dabei durch mindestens drei verschiedene Kanten beschrieben, die den Rand dieser Fläche darstellen. 
Die Fläche um den Graphen herum, welche scheinbar unbegrenzt groß ist, wird äußerstes Gebiet genannt.
\section{Doubly connected edge list}
Um planare Graphen ohne Informationsverlust zu speichern werden in der Informatik Referenzen zwischen den einzelnen Bestandteilen des Graphen eingesetzt. \\
In der sogenannten \q{Doubly connected edge list} (DCEL) erhält eine Kante, die aus einem Anfangsknoten und Endknoten besteht jeweils eine Vorgänger-, eine Nachfolger- und eine Zwillingskante. 
Jedem Knoten wird außerdem eine ausgehende Kante und allen Flächen eine anliegende Kante zugewiesen. \\
Diese zwischenobjektlichen Referenzierungen ermöglichen es, ausgehend von einem Element ohne umfangreiche Berechnungen auf alle anderen Objekte zu schließen, indem bei Knoten und Flächen die zugehörigen Kanten, beziehungsweise bei Kanten deren Vorgägner und Nachfolger betrachtet werden.
\todoinline{Johann wollte DCEL noch mal überarbeiten}
\section{AutoCAD}
%Konstruktionsprogramm statt architektenprogramm
AutoCAD ist ein grafischer Zeichnungseditor, welcher zum Erstellen von technischen Zeichnungen und dem Modellieren von Objekten verwendet wird.
AutoCAD verwendet dabei einfache Objekte wie Linien, Kreise und Bögen, um auf deren Grundlage kompliziertere Objekte zu erschaffen.
Zu AutoCAD gehörig wurden das Dateiformat \q{.dxf} entwickelt, welches als Industriestandard zum Austausch von CAD-Dateien dient. \\
Der Grundriss, welcher als Ausgangspunkt dieser Arbeit fungiert, ist in AutoCAD erstellt wurden und wird dem zu erstellenden Programm in Form einer .dxf-Datei bereitgestellt.
\section{OpenSCAD}
OpenSCAD ist eine freie CAD-Modellierungssoftware, welche auf Basis einer textbasierten Beschreibungssprache 3D-Modelle erzeugt.
OpenSCAD bietet dabei verschiedene Vorteile während des Modellierungsvorganges, wie beispielsweise dem farbigen Hervorheben oder der Modularisierung bestimmter Objekte. \\
%Modellierung
Die Modellierung von einfachen Basisobjekten in OpenSCAD erfolgt durch das Verwenden von bestimmten Schlüsselwörtern wie \icode{cube()}, \icode{sphere()} oder \icode{cylinder()} und dem anschließenden Übergeben von Parametern in Klammern.
Diese Basisobjekte werden anschließend durch Mengenoperationen wie Vereinigungen (\icode{union()}), Differenzen (\icode{difference()}) oder Überschneidungen (\icode{intersection()}) und Transformationen wie Skalierungen (\icode{scale()}), Rotationen (\icode{rotate()}) oder Translationen (\icode{translate()}) mit einander verknüpft und kombiniert, um neue und komplexere Objekte nach eigenen Ansprüchen zu erhalten.
Neben solchen einfachen Objekten, wird außerdem die Möglichkeit geboten, komplexere Objekte wie Polygone (\icode{polygon()}) zu erstellen und diese dann ausgehend vom zweidimensionalen Polygon in ein dreidimensionales Polygon umzuwandeln (\icode{linear\_extrude()}), welches vor allem das Umwandeln von komplexen Formen in Objekte erleichtert. \\
Die Anweisungen, welche OpenSCAD zum Modellieren verwendet, werden in einfachen Textdateien im \q{.scad}-Format gespeichert.
Die Simplizität dieser Textdateien erlaubt es, die aus dem Programm erhaltenen Anweisungen in .scad-Dateien zu speichern, welche dann von OpenSCAD eingelesen, eingesehen und bearbeitet werden können. \\
%Drucken
Die Modelle, die so mit OpenSCAD erstellt wurden, können anschließend mit dem 3D-Drucker ausgedruckt werden.
Dazu werden die Modelle in Dateien des \q{.stl}-Formats konvertiert, welche schlussendlich mittels der dem 3D-Drucker beiliegenden Software entweder durch einen USB-Anschluss des 3D-Druckers oder auf einer SD-Karte gespeichert ausgedruckt werden.
Auch an dieser Stelle des Modellierungsvorganges bietet sich OpenSCAD wieder an, da es von Haus aus die Option zur Konvertierung vom .scad-Format zu .stl mitliefert.
\section{3D-Drucker MakerBot X$>$9000}
%Erklärung 3D-Drucker
Der vorliegende 3D-Drucker ist das Modell XYZ der Firma MakerBot.
Dieser verfügt über eine höhenverstellbare Grundplatte, auf der das Filament aufgetragen und so auch das finale Objekt gedruckt wird, und einen sogenannten \q{Extruder}, welcher die Funktion übernimmt, das Filament mit dem gedruckt werden soll, zu erhitzen und kontrolliert auf die Grundplatte bzw. das gedruckte Objekt aufzutragen. \\
Die Höhe der Grundplatte wird automatisch vom Drucker variiert und nach dem Ende jedes Druckvorganges wieder komplett nach unten gefahren.
Um die Beweglichkeit des Extruders zu garantieren, ist dieser so auf drei Achsen befestigt, dass drei Motoren ihn darauf verschieben können. \\
Abhängig vom Filament bzw. der Temperatur, bei der dieses aufgetragen wird, der Bewegungsgeschwindigkeit des Extruders und der Filamentstärke, die der Extruder aufträgt, lässt sich die gewünschte Druckqualität anpassen.
Eine niedrige Qualität ist dabei in den meisten Fällen mit einer erheblich kürzeren Druckzeit verbunden. \\
Die Druckzeit wird außerdem von der eingestellten Ausfüllung von geschlossenen Objekten beeinflusst.
So kann man beispielsweise Quader nicht komplett mit Filament füllen lassen, sondern beispielsweise mit einem Bienenwabenmuster durchsetzen, sodass nur ein geringer Teil des Objektes ausgefüllt wird, aber dennoch Stabilität garantiert ist.
Indem so also ein stark verringerter Betrag an Filament aufgetragen werden muss, wird auch die Druckzeit reduziert.