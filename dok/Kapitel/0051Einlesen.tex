\chapter{Vorgehen zur Problemlösung}
\section{Einlesen des Grundrisses}
\subsection{Funktionsweise der Bibliothek \textit{kabeja}}
Den Beginn der Verarbeitung markiert hierbei die Grundrissdatei, in welcher sämtliche Werte, welche im weiteren Verlauf des Programmes relevant werden, enthalten sind.
Das Einlesen der Daten eines Grundrisses, wie in Abb. 5, erfolgt mit der Java-Bibliothek „kabeja“. 
Diese ermöglicht es, aus .dxf-Dateien alle DXF-Objekte eines bestimmten Typs zu erhalten und deren Werte in einer Liste zu speichern und später zu verarbeiten \cite{kabeja}.
\begin{code}[DXF File Parser]
public static ArrayList<Line> getAutocadFile(String filePath) throws ParseException {
	ArrayList<Line> vcs = new ArrayList<>();
	Parser parser = ParserBuilder.createDefaultParser();
	parser.parse(filePath, DXFParser.DEFAULT_ENCODING);
	DXFDocument doc = parser.getDocument();
	List lst = doc.getDXFLayer("0").getDXFEntities(DXFConstants.ENTITY_TYPE_LINE);
	for (int index = 0; index < lst.size(); index++) {
		DXFLine dxfline = (DXFLine) lst.get(index);
		Line v = new Line(
		new Vector(round2(dxfline.getStartPoint().getX()), round2(dxfline.getStartPoint().getY())),
		new Vector(round2(dxfline.getEndPoint().getX()), round2(dxfline.getEndPoint().getY())));
		vcs.add(v);
	}
	
	return vcs;
}
\end{code}
\begin{Bild}{Eins Grafik}
	\includegraphics[scale=0.1]{example-image-a}
\end{Bild}
In dieser Anwendung wird eine Funktion der Klasse \q{DXFReader} verwendet, welche den Pfad zur .dxf-Datei als Parameter übergeben bekommt. 
Aus dieser Datei werden dann alle DXF-Objekte, die mit dem Typen \icode{DXFLine} übereinstimmen, in einer Liste zurückgegeben. 
Die Koordinaten der Start- und Endpunkte der DXFLines  in dieser Liste werden anschließend in eine Liste von Linien übertragen, welche im weiteren Programmablauf unter anderem für die Umwandlung des Graphen in die DCEL verwendet werden.
\subsection{Funktionsweise der GUI}
\todo{asdf}