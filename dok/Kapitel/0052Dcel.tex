\section{Erstellen der Doubly-connected Edge List}
\subsection{Graph Klasse}
Die wichtigste Klasse der Anwendung ist \icode{Graph}. 
Sie verbindet alle grafischen Bestandteile und kann mit einer übergebenen Liste aus Linien mittels mehrerer Zwischenoperationen eine vollständige DCEL, also einen planaren Graph erstellen.
Dafür sind drei Listen vom Typ \icode{Edge}(Kante), \icode{Node}(Knoten) und \icode{Face}(Fläche) gespeichert, die eine DCEL repräsentieren.

\subsection{Line-to-Edge Konvertierung}
\label{subsec:ltoe} 
Initialisierend werden aus den Koordinaten der eingelesenen Start- und Endpunkte in Knoten umgewandelt.
Aus diesen noch unreferenzierten Knoten definiert man dann die Kanten mit entsprechendem Start- und Endknoten.
Um für die Eindeutigkeit der Knoten zu sorgen, gibt die Funktion \icode{createNode()} ein äquivalenten Knoten zu einem Punkt zurück und fügt ihn zu der Liste von Knoten hinzu, wenn er noch nicht in ihr vorhanden ist.

\begin{code}[Line-to-Edge Konvertierung]
private void processData(ArrayList<Line> ls) {
	for (Line l : ls) {
		edges.add(new Edge(createNode(l.getP1()), createNode(l.getP2())));
	}
}
\end{code}

\begin{code}[\icode{createNode()} Funktion]
private Node createNode(Vector p) {
	for (Node n : nodes) {
		if (n.getOrigin().equals(p)) {
			return n;
		}
	}
	nodes.add(new Node(p));
	return (nodes.get(nodes.size() - 1));
}
\end{code}

\todoinline{Hier könnte man noch was zu sagen}

\subsection{Zwillingskantengenerierung}
Weiterhin werden für sämtliche Kanten die invertierten Gegenstücke, auch als \q{Zwillingskanten} bezeichnet, durch Vertauschen der Start- und Endknoten gebildet.
Direkt nach dem Hinzufügen der Zwillingskante wird jeweils eine Referenz erstellt, die in beiden Kanten auf den jeweils zugehörigen Zwilling verweist. 
In der Kantenliste existiert nun für jede Linie die der DXF-Reader eingelesen hat, zwei zueinander komplementäre Kanten.
Die Zwillingskanten dienen dem Aufbau der DCEL, um eine Iteration in beide Richtungen um die Flächen und so auch eine einfachere Umwandlung zu ermöglichen.

\begin{code}[Hinzufügen der Zwillingskanten]
private void computeTwins() {
	int amount = edges.size();
	for (int i = 0; i < amount; i++) {
		edges.add(edges.get(i).generateTwin());
		edges.get(i).setTwin(edges.get(edges.size() - 1));
		edges.get(edges.size() - 1).setTwin(edges.get(i));
	}
}
\end{code}

\subsection{Nachfolger- und Vorgängerermittlung}
Zum Erstellen der Referenzen werden zuerst alle ausgehenden Kanten der Knoten, das heißt die den jeweiligen Knoten als ihre Startkante besitzen, in einer zweidimensionalen \icode{ArrayList} gespeichert.
Die erste Dimension steht für den Index des Knotens in der erstellten Knotenliste.
In der zweiten Dimension liegen die den Indizes zugehörigen ausgehenden Kanten vor.
Da diese durch eine \icode{for()} Schleife mit der oben stehenden Bedingung aufgerufen werden, liegen die Kanten im \icode{Array} in zufälliger Reihenfolge vor.
Sortiert werden die Kanten mithilfe des Winkels, den die Funktion \icode{atan2()} der \icode{java.Math}-Bibliothek liefert, im positiven Drehsinn.
Begonnen wird dabei mit dem kleinsten zurückgegebenen Winkel.
Daraus ergibt sich, dass das vorherige Element die Kante, die mit einem größeren Winkel \q{links} der Betrachteten darstellt, beziehungsweise das nachfolgende Element, die mit einem kleineren Winkel \q{rechts} der Ausgewählten liegt darstellt. \\
Für jede ausgehende Kante \icode{e} können nun folgende Referenzen gesetzt werden:

\begin{itemize}
	\item Das vorherige Element der \icode{ArrayList} stellt den Nachfolger der Zwillingskante von \icode{e} dar.
	\item Der Vorgänger von \icode{e} ist der Zwilling der in der \icode{ArrayList} nachfolgenden Kante.
\end{itemize}

Genannte Referenzen werden nun gesetzt, sodass die Verknüpfungen zwischen den Kanten fertiggestellt sind.\\
\todoinline{code eventuell bilder definitiv}

\subsection{Flächenerstellung}
Mithilfe einer Schleife können die einzelnen Flächen erschlossen werden.
Es muss lediglich fortlaufend alle Nachfolger einer gewählten Kante gesucht werden bis die gefundene Kante einen Nachfolger darstellt, bis alle Kanten genau einmal behandelt wurden.
Dafür wurde eine repräsentative \icode{Boolean-ArrayList} mit der Länge der Kantenliste erstellt, die mit ihren Werten den Status der Kanten darstellt.
Folglich besteht diese am Anfang nur aus \icode{false} Werten die während der Schleife alle auf \icode{true} gesetzt werden.
Die Speicherung der Flächen folgt dem DCEL-Standard, sodass nur eine anliegende Kante zugewiesen wird, mit der man, aufgrund von den Referenzen alle Kanten und Knoten der Fläche ermitteln kann.
Das äußere Gebiet des planaren Graphen lässt sich an der Umlaufrichtung der Kanten erkennen.
Es ist die einzige Fläche, die im mathematisch negativen Drehsinn ausgerichtet ist.
Unterschieden wird sie im weiteren Programmablauf mithilfe der Gaußschen Trapezformel.
Bei negativer Umlaufrichtung ist das Ergebnis dieser negativ.\\
\todoinline{Bitte noch mal überarbeiten, total wirr}

\subsection{Vervollständigung der Knoten}
Die letzte nötige Operation ist die Speicherung einer anliegenden Edge in den Nodes.\\
\todoinline{man könnte das im code auch vorher tun denke ich}