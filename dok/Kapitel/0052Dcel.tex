\section{Erstellen der DCEL}
\subsection{Line-to-Edge Konvertierung}
In der Hauptklasse des Programms wird aus der Liste von Lines ein planarer Graph erstellt. 
Dabei werden die Start- und Endpunkte der eingelesenen Lines in Nodes ohne eine Referenz auf eine anliegende Edge umgewandelt und diese als Start- und Endnode der entsprechenden Edge gesetzt.
Um für die Eindeutigkeit der Nodes zu sorgen, werden nur für Koordinaten neue Nodes erstellt, die zuvor in den Iteration noch nicht erschienen sind.(Funktion \icode{createNode()})

\begin{lstlisting}[caption = {Line-to-Edge Konvertierung}]
	private void processData(ArrayList<Line> ls) {
		for (Line l : ls) {
			Node n1 = createNode(l.getP1());
			Node n2 = createNode(l.getP2());
			edges.add(new Edge(n1, n2));
		}
	}
\end{lstlisting}
\subsection{Twin-Edge Generierung}
Um aus diesen Edges die invertierten Gegenstücke, auch als \q{Twinedges} bezeichnet, zu erhalten, werden alle Edges, die in der Liste bereits vorhanden sind, betrachtet und neue Edges hinzugefügt, die im Vergleich zu den ursprünglichen Edges vertauschte Start- und Endnodes besitzen.
Direkt nach dem Hinzufügen der neuen Edge wird jeweils eine Referenz erstellt, die in beiden Edges auf den jeweils zugehörigen Zwilling verweist. 
In der Liste der Edges existiert nun für jede Line die der DXF--Reader eingelesen hat, zwei zueinander komplementäre Edges.
\subsection{Nachfolger- und Vorgängerermittlung}
Die in der DCEL besagten Referenzen von einer Edge zu ihrem Vorgänger- und Nachfolger werden durch eine Sortierung der ausgehenden Edges an den Nodes realisiert.
\subsubsection{Winkelberechnung an den Knoten}
Dabei wird zuerst eine ArrayList of ArrayList of Double erstellt, die die jeweiligen \icode{atan2()} Winkel im mathematisch positiven Sinn enthalten. \todo{schreiben...}
\subsubsection{Setzen der Referenzen}
Den entstandenen sortierten ArrayLists kann entnommen werden, wie die Edges an einem Node angeordnet sind.
Da nur ausgehende Edges betrachtend werden, können für jede Edge und ihren Twin einzeln Referenzen gesetzt werden.\\ 
Es ergibt sich bei Betrachtung einer Edge:
\begin{itemize}
	\item Die in der ArrayList vorgängige Edge ist der Nachfolger der Twin Edge
	\item Der Twin der in der ArrayList vorgängigen Edge ist der Vorgänger der betrachteten Edge
\end{itemize}

\subsection{Flächenerstellung}

\subsection{Vervollständigung der Knoten}