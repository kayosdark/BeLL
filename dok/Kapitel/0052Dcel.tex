\section{Erstellen der DCEL}
\subsection{Graph Klasse}
Die wichtigste Klasse der Anwendung ist \icode{Graph}. 
Sie verbindet alle grafischen Bestandteile und kann mit einer übergebenen Liste aus Linien mittels mehrerer Zwischenoperationen eine vollständige DCEL, also einen planaren Graph erstellen.
Dafür sind drei Listen vom Typ \icode{Edge}(Kante), \icode{Node}(Knoten) und \icode{Face}(Fläche) gespeichert, die eine DCEL repräsentieren.
\subsection{Line-to-Edge Konvertierung}
\label{subsec:ltoe} 
Initialisierend werden die Start- und Endpunkte der eingelesenen Linien in Knoten ohne eine Referenz auf eine anliegende Kante umgewandelt und diese als Start- und Endknoten der entsprechenden Kante gesetzt.
Um für die Eindeutigkeit der Knoten zu sorgen, gibt die Funktion \icode{createNode()} ein äquivalenten Knoten zu einem Punkt zurück und fügt ihn zu der Liste an Knoten hinzu, wenn er noch nicht in ihr existiert.

\begin{code}[Line-to-Edge Konvertierung]
private void processData(ArrayList<Line> ls) {
	for (Line l : ls) {
		edges.add(new Edge(createNode(l.getP1()), createNode(l.getP2())));
	}
}
\end{code}
\begin{code}[\icode{createNode()} Funktion]
private Node createNode(Vector p) {
	for (Node n : nodes) {
		if (n.getOrigin().equals(p)) {
			return n;
		}
	}
	nodes.add(new Node(p));
	return (nodes.get(nodes.size() - 1));
}
\end{code}
\todoinline{Hier könnte man noch was zu sagen}
\subsection{Zwillingskantengenerierung}
Um aus den entstandenen Kanten die invertierten Gegenstücke, auch als \q{Zwillingskanten} bezeichnet, zu erhalten, werden alle Kanten, die in der Liste bereits vorhanden sind, betrachtet und neue Kanten hinzugefügt, die im Vergleich zu den ursprünglichen Kanten vertauschte Start- und Endknoten besitzen.
Direkt nach dem Hinzufügen der neuen Kante wird jeweils eine Referenz erstellt, die in beiden Kanten auf den jeweils zugehörigen Zwilling verweist. 
In der Kantenliste existiert nun für jede Linie die der DXF-Reader eingelesen hat, zwei zueinander komplementäre Kanten.
\begin{code}[Hinzufügen der Zwillingskanten]
private void computeTwins() {
	int amount = edges.size();
	for (int i = 0; i < amount; i++) {
		edges.add(edges.get(i).generateTwin());
		edges.get(i).setTwin(edges.get(edges.size() - 1));
		edges.get(edges.size() - 1).setTwin(edges.get(i));
	}
}
\end{code}
\subsection{Nachfolger- und Vorgängerermittlung}
Das Erstellen der Referenzen werden zuerst alle ausgehenden Edges der Nodes, das heißt alle Edges, die den jeweiligen Node als ihren Startnode besitzen, in einer zweidimensionalen ArrayList gespeichert.
Die erste Dimension steht für den Index des Nodes in der erstellten Nodeliste für den in der zweiten Dimension die jeweiligen ausgehenden Edges vorliegen.
Da diese durch eine \icode{for()} Schleife mit der oben stehenden Bedingung herausgesucht werden, sind die Edges im Array in zufälliger Reihenfolge, also nicht nach der Anordnung am Node gegenwärtig.
Jetzt werden die Edges anhand des \icode{atan2()} Winkel am vorliegenden Node im mathematisch positiven Drehsinn sortiert.
Daraus ergibt sich, dass das vorherige bzw. nachfolgende Element einer Edge die Edge, die \q{links} bzw. \q{rechts} der Betrachteten liegt darstellt.\\
Für jede ausgehende Edge \icode{e} können nun folgende Referenzen gesetzt werden:
\begin{itemize}
	\item Das vorherige bzw. letzte Element der ArrayList, wenn die betrachtete Edge den Index 0 hat, stellt den Nachfolger der Twinedge von \icode{e} dar.
	\item Der Twin der in der ArrayList nachfolgenden Edge bzw. ersten Edge, wenn die betrachtete Edge das letzte Element ist, ist der Vorgänger von \icode{e}
\end{itemize}
Genannte Referenzen werden nun gesetzt, sodass die Verknüpfungen zwischen den Edges fertiggestellt sind.
\todoinline{code eventuell bilder definitiv}
\subsection{Flächenerstellung}
Durch eine Schleife können die einzelnen Flächen herausgefiltert werden.
Zuerst wird eine Boolean-ArrayList mit der selben Länge der Edgeliste erstellt, welche die Indices der schon behandelten Kanten repräsentiert.
Folglich besteht die Liste anfangs nur aus \icode{false} Werten.
Jetzt wird fortlaufend eine nicht behandelte Kante herausgesucht und von dieser   \todoinline{keene Ahnung wie ich das mit der repräsentativen Liste formulieren soll}

Ab diesem Schritt existieren alle Flächen vollständig in der Flächenliste.
Das unendliche bzw. äußere Gebiet lässt sich 


\subsection{Vervollständigung der Knoten}
Die letzte nötige Operation ist die Speicherung einer anliegenden Edge in den Nodes.\todoinline{man könnte das im code auch vorher tun denke ich}