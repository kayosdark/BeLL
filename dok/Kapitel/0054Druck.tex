\section{Druckvorgang}
%\todoinline{Beschreibung, warum die Bauteile auf- und verteilt werden müssen}
%genaue Maße der Druckplatte; siehe: 
%\verb|https://eu.makerbot.com/fileadmin/Inhalte/Support/Manuals/German_UserManual_V.4_Replicator2.pdf|
%oder Kapitel 4 unten
% [28.5 x 15.3 x 15.5 cm]
Nachdem die einzelnen Bestandteile des Modells berechnet wurden, können diese gedruckt werden.
Hierbei ist allerdings zu bedenken, dass alle einzelnen Bauteile nicht in einem Druckvorgang gedruckt werden können, da die Grundplatte des 3D-Druckers zu klein ist. \\
In unserem Fall wies diese eine Länge von 28,5 cm und eine Breite von 15,2 cm auf (siehe Quelle \cite{makerbotspecs}).
Um die Dimensionen der Grundplatte effektiv auszunutzen, unterteilen wir die Grundplatten, welche aufgrund ihrer Größe nicht in einem Durchgang druckbar sind.
Außerdem werden einzelne Elemente des Modells zu Objektgruppen zusammengefasst, die nebeneinander gedruckt werden können.

\subsection{Aufteilen der Grundplatten}
\todoinline{Beschreiben, wie das gemacht wird}

\subsection{Verteilung der Bauteile}
\todoinline{Beschreibung der Verteilung einzelner Bauteile für passenden Druckvorgang}