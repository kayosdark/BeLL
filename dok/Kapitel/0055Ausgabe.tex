\section{Ausgabe der Resultate}
Nachdem die Berechnungen des Modells abgeschlossen ist, müssen die Ergebnisse nun ausgegeben werden.
Für die Ausgabe der Dateien wird ein Ordner in dem Verzeichnis erstellt, in dem auch das Programm ausgeführt wird.

\subsection{Erstellung des Ausgabeordners}
Als Name des Ordners wird der Name genutzt, den der Nutzer in der GUI eingegeben hat.
Im Unterordner \q{scad} dieses Ordners finden sich dann sämtliche .scad-Dateien, im Unterordner \q{stl} alle .stl-Dateien. \\
Zum Erstellen der Unterordner wird die Funktion \icode{createDirs()} verwendet (vgl. Codeausschnitt~6).
Diese legt mittels der Anweisung \icode{Files.createDirectories()} die Unterverzeichnisse an.
Sollte der übergeordnete Ausgabeordner noch nicht vorhanden sein, wird er durch die Anweisung direkt mit angelegt. \\
 
\begin{code}[Erstellung der Unterordner]
	private static void createDirs(String folderName) {
		try {
			Files.createDirectories(Paths.get("./" + folderName + "/scad"));
			Files.createDirectories(Paths.get("./" + folderName + "/stl"));
		} catch (IOException e) {
			e.printStackTrace();
		}
	}
\end{code}

\subsection{STL Konvertierung}
Um die ermittelten .scad-Dateien im .stl-Format auszugeben, wird die Kommandozeilenfunktionalität von OpenSCAD verwendet.
Hierzu werden zunächst alle Dateien im \q{scad}-Unterordner mit der Endung .scad durch die Klasse \icode{SCADFinder} ermittelt und in einem Feld ausgegeben.
Dies geschieht mittels der Funktion \icode{findFiles()} (vgl. Codeausschnitt~7), welche durch einen \icode{FileFilter} alle Dateien mit gewünschter Endung zurückgibt. \\

\begin{code}[Ausgabe aller .scad-Dateien aus einem Ordner]
	public static File[] findFiles(String folderName){
		File dir = new File(".\\"+folderName+"\\scad\\");
		
		return dir.listFiles(new FilenameFilter() { 
			public boolean accept(File dir, String filename)
			{ return filename.endsWith(".scad"); }
		});
	}
\end{code}

Mit den Namen dieser Dateien als Argumente, kann im Anschluss die Umwandlung ins .stl-Format stattfinden.
Die verwendete Konsolenanweisung setzt sich aus dem Pfad zur \icode{openscad.exe}, dem Parameter \icode{-o} und zwei weiteren Parametern zusammen. 
Die beiden weiteren Parameter stehen zum einen für den Namen der Ausgabedatei und zum anderen für den Namen der Eingabedatei. \\
Eine vollständige Anweisung könnte wie folgt aussehen: \icode{C:\textbackslash \textbackslash Program Files\textbackslash OpenSCAD\textbackslash openscad.exe -o \textbackslash stl\textbackslash walls1.stl \textbackslash scad\textbackslash walls1.scad} .
Diese Anweisungen verwendet die Klasse \icode{STLConverter} mittels eines \icode{ProcessBuilder}, um die .stl-Dateien im \q{stl}-Unterordner zu erstellen (vgl. Codeausschnitt~8). \\

\begin{code}[Ausführung der Kommandozeilenanweisungen]
	public static void convert (String fileName, String folderName) throws InterruptedException {
		Process p;
		ProcessBuilder b;
		try {
			b = new ProcessBuilder(/*Anweisung*/);
			p = b.start();
						
			p.waitFor();
		} catch (IOException e) {
			e.printStackTrace();
		}
	}
\end{code}